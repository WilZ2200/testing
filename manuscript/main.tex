\documentclass[12pt,letterpaper]{article}

% Mathematics
\usepackage{amsmath}
\usepackage{amssymb}

% Tables
\usepackage{booktabs}
\usepackage{threeparttable}
\usepackage{multirow}
\usepackage{array}
\usepackage{longtable}
\usepackage{tabularx}

% Graphics
\usepackage{graphicx}
\usepackage{float}
\usepackage{caption}
\usepackage{subcaption}

% Layout
\usepackage[margin=1in]{geometry}
\usepackage{setspace}
\usepackage{lineno}
\usepackage{pdflscape}

% References and links
\usepackage{natbib}
\usepackage[colorlinks=true,linkcolor=blue,citecolor=blue,urlcolor=blue]{hyperref}

% Other
\usepackage{xcolor}

% Line numbering and double spacing
\linenumbers
\doublespacing

% Title information
\title{\textbf{Patient Preferences for Breast Cancer Screening: A Discrete Choice Experiment}}

\author{
  Author One\textsuperscript{1,*} \and
  Author Two\textsuperscript{2} \and
  Author Three\textsuperscript{1}
}

\date{
  \textsuperscript{1}Department of Health Policy and Management, University Name \\
  \textsuperscript{2}Department of Economics, University Name \\[1em]
  \textsuperscript{*}Corresponding author: author.one@university.edu \\[2em]
  \today
}

\begin{document}

\maketitle

\begin{abstract}
\noindent
\textbf{Background:} Breast cancer screening guidelines continue to evolve, yet the preferences of women regarding screening attributes remain insufficiently understood. Incorporating patient preferences into screening program design is essential for improving uptake and aligning clinical practice with patient-centered care.

\noindent
\textbf{Objective:} This study employed a discrete choice experiment (DCE) to estimate women's preferences for breast cancer screening attributes, calculate willingness-to-pay (WTP) estimates, and identify preference heterogeneity across population subgroups.

\noindent
\textbf{Methods:} A DCE was administered to 500 women aged 40--74 recruited via an online panel and clinic-based sampling. The experiment included seven attributes: screening method, frequency, out-of-pocket cost, sensitivity, false-positive rate, waiting time for results, and physical discomfort. Choice data were analyzed using conditional logit, mixed logit, and latent class models.

\noindent
\textbf{Results:} All screening attributes significantly influenced women's choices. Sensitivity and out-of-pocket cost were the most important attributes. Women were willing to pay approximately \$180 for MRI over mammography, \$250 for a 25-percentage-point increase in sensitivity, and \$80 per 5-percentage-point reduction in false-positive rates. Latent class analysis identified three distinct preference segments: accuracy-focused (40\%), cost-conscious (35\%), and convenience-seeking (25\%) women.

\noindent
\textbf{Conclusions:} Women exhibit substantial and heterogeneous preferences for breast cancer screening attributes. Screening programs that offer flexible options tailored to distinct preference segments may improve uptake and patient satisfaction. These findings have implications for screening guideline development and coverage policy decisions.

\vspace{1em}
\noindent
\textbf{Keywords:} discrete choice experiment; breast cancer screening; patient preferences; willingness-to-pay; latent class analysis; mammography; health economics

\vspace{1em}
\noindent
\textbf{JEL Classification:} I11, I12, I18, C25
\end{abstract}

\newpage
\tableofcontents
\newpage

% Introduction
\section{Introduction}
\label{sec:introduction}

Breast cancer remains the most commonly diagnosed malignancy among women worldwide, accounting for approximately 2.3 million new cases and 685,000 deaths annually \citep{sung2021global, WHO2024breast}. In the United States alone, the American Cancer Society estimates that over 310,000 new cases of invasive breast cancer will be diagnosed in 2024, with approximately 42,000 women expected to die from the disease \citep{siegel2024cancer}. The lifetime risk of developing breast cancer for American women is approximately one in eight, making it a pervasive public health concern that touches virtually every community \citep{ACS2024facts}. Despite significant advances in treatment modalities over the past several decades, early detection through systematic screening remains one of the most effective strategies for reducing breast cancer mortality, as tumors identified at earlier stages are associated with substantially better prognoses and less aggressive treatment requirements \citep{berry2005effect, tabar2003mammography}.

The centrality of screening to breast cancer control has motivated the development of organized screening programs across most high-income countries. Mammography has served as the primary screening modality for over four decades and is widely regarded as the gold standard for population-based breast cancer detection \citep{marmot2013benefits, nelson2009screening}. Randomized controlled trials conducted throughout the 1970s and 1980s demonstrated that regular mammographic screening can reduce breast cancer mortality by 15--30\% among women aged 50--74 \citep{nystrom2002long, gotzsche2013screening}. More recently, emerging technologies have expanded the screening landscape. Digital breast tomosynthesis (3D mammography) has shown improved cancer detection rates and reduced recall rates compared with conventional digital mammography \citep{friedewald2014breast}. Magnetic resonance imaging (MRI) offers superior sensitivity, particularly for women with dense breast tissue or elevated genetic risk, though at considerably higher cost and with greater rates of false-positive findings \citep{lord2007systematic, saslow2007american}. Supplemental ultrasound screening has also been investigated as an adjunct modality for women with dense breast tissue, demonstrating the ability to detect additional cancers missed by mammography alone \citep{berg2012detection, ohuchi2016sensitivity}.

Despite the demonstrated benefits of breast cancer screening, the optimal screening strategy remains a subject of vigorous scientific and policy debate. Screening guidelines issued by major medical organizations have diverged in their recommendations regarding the age at which screening should commence, the frequency of screening, and the modalities to be employed. The United States Preventive Services Task Force (USPSTF) issued a landmark updated recommendation in 2024, lowering the recommended starting age for biennial mammographic screening from 50 to 40 years for women at average risk \citep{USPSTF2024breast}. This revision represented a significant departure from the USPSTF's 2016 recommendation, which had designated screening for women aged 40--49 as an individual decision to be made in consultation with a clinician \citep{siu2016screening}. The American Cancer Society, by contrast, has recommended annual mammographic screening beginning at age 45 since 2015, with the option to begin at age 40 and transition to biennial screening at age 55 \citep{oeffinger2015breast}. International guidelines exhibit even greater variation: the United Kingdom's National Health Service (NHS) Breast Screening Programme invites women aged 50--70 for triennial mammographic screening \citep{marmot2013benefits}, while the European Commission Initiative on Breast Cancer has recommended biennial screening for women aged 50--69 with consideration of extension to ages 45--74 \citep{defined_europerecs2022}.

Underlying these guideline disagreements are fundamental tensions between the benefits and harms of screening. While mammographic screening can detect cancers at earlier, more treatable stages, it also carries well-documented risks including false-positive results that lead to unnecessary biopsies and psychological distress, radiation exposure from repeated imaging, and---most controversially---overdiagnosis \citep{welch2016breast, bleyer2012effect}. Overdiagnosis refers to the detection of cancers that would never have become clinically significant during a woman's lifetime, leading to overtreatment with its attendant physical, psychological, and financial costs \citep{puliti2012overdiagnosis, jorgensen2009overdiagnosis}. Estimates of the magnitude of overdiagnosis in mammographic screening vary widely, from less than 5\% to over 50\% depending on the methodology employed, contributing to ongoing disagreement about the net benefit of screening in certain age groups \citep{biesheuvel2007policing, etzioni2013overdiagnosis}. These trade-offs between benefits and harms underscore the fact that breast cancer screening decisions are inherently preference-sensitive---that is, the optimal choice depends in part on how individual women weigh the potential advantages against the potential disadvantages of screening.

The recognition that screening decisions are preference-sensitive has coincided with a broader movement toward patient-centered care in medicine. Shared decision-making, in which clinicians and patients collaborate to make healthcare decisions that reflect both clinical evidence and patient values, has become an increasingly prominent paradigm in clinical practice and health policy \citep{barry2012shared, elwyn2012shared}. Multiple professional organizations now recommend shared decision-making for breast cancer screening, particularly for women in age groups where the balance of benefits and harms is most uncertain \citep{USPSTF2024breast, oeffinger2015breast}. Implementing shared decision-making effectively, however, requires an understanding of what women actually value in breast cancer screening programs. While qualitative studies have explored women's attitudes toward screening \citep{hersch2013women, waller2014understanding}, there remains a need for rigorous quantitative evidence on the relative importance women place on different screening attributes and how they make trade-offs among competing considerations.

Discrete choice experiments (DCEs) have emerged as the preferred stated preference methodology for eliciting and quantifying individual preferences in health economics and health services research \citep{ryan2001using, lancsar2008conducting, debbekkergrob2012dcereview}. Grounded in Lancaster's characteristics theory of value \citep{lancaster1966new} and McFadden's random utility theory \citep{mcfadden1974conditional}, DCEs present respondents with a series of hypothetical choice scenarios in which they select their preferred option from a set of alternatives described by systematically varied attributes. By observing how individuals trade off between different attribute levels, researchers can estimate the marginal utility associated with each attribute, calculate willingness-to-pay (WTP) values when a cost attribute is included, and identify heterogeneity in preferences across population subgroups \citep{train2009discrete, hensher2015applied}. The DCE methodology has been widely applied in health economics to study preferences for a diverse range of health services, treatments, and programs \citep{clark2014discrete, soekhai2019discrete}, and a growing body of literature has employed DCEs to investigate preferences for cancer screening specifically \citep{marshall2010conjoint, de2012labels}.

Although previous DCE studies have examined preferences for breast cancer screening, several important gaps persist in the literature. Many existing studies have focused on a limited set of screening attributes, frequently omitting attributes related to the screening experience such as physical discomfort and waiting time for results, which qualitative research suggests are important to women \citep{whelehan2013review, nelson2020factors}. Furthermore, the majority of prior studies have relied exclusively on conditional logit models, which assume homogeneous preferences across the sample population and impose the restrictive independence of irrelevant alternatives (IIA) assumption \citep{hausman1984specification}. Relatively few studies have employed more flexible modeling approaches such as mixed logit or latent class models that can accommodate and characterize preference heterogeneity \citep{hole2008modelling, greene2003latent}. Finally, the rapidly evolving landscape of screening guidelines---particularly the USPSTF's 2024 recommendation change---motivates renewed investigation of how women's preferences align with or diverge from expert-recommended screening strategies.

The present study addresses these gaps by conducting a comprehensive DCE to elicit women's preferences for breast cancer screening. Specifically, this study has three objectives. First, we estimate the relative importance of seven screening attributes---screening method, screening frequency, out-of-pocket cost, test sensitivity, false-positive rate, waiting time for results, and physical discomfort---using a conditional logit model. Second, we calculate WTP estimates for improvements in each attribute to provide policy-relevant valuations denominated in monetary terms. Third, we employ mixed logit and latent class models to identify and characterize preference heterogeneity, testing whether distinct subgroups of women exhibit systematically different preference patterns that may warrant differentiated screening approaches.

The remainder of this paper is organized as follows. Section~\ref{sec:literature_review} provides a comprehensive review of the relevant literature on breast cancer screening guidelines, patient preferences in cancer screening, DCE methodology, and previous DCE applications in breast cancer screening. Section~\ref{sec:methods} details the study's methodological framework, including the theoretical foundation, experimental design, survey administration, and econometric specifications. Section~\ref{sec:results} presents the empirical results from the conditional logit, mixed logit, and latent class models, along with WTP estimates and subgroup analyses. Section~\ref{sec:discussion} discusses the findings in the context of the existing literature and considers their implications for clinical practice and policy. Section~\ref{sec:conclusion} concludes with a summary of key contributions and recommendations for future research.


% Literature Review
\section{Literature Review}
\label{sec:literature_review}

This section provides a comprehensive review of the literature relevant to the present study. We begin by examining the current evidence base and guidelines for breast cancer screening, followed by a discussion of patient preferences in the context of cancer screening. We then review the DCE methodology and its applications in health economics before summarizing previous DCE studies focused specifically on breast cancer screening. The section concludes by identifying the research gap that this study addresses.

\subsection{Breast Cancer Screening: Current Evidence and Guidelines}
\label{subsec:screening_evidence}

The evidence base for breast cancer screening has been built primarily upon a series of randomized controlled trials (RCTs) conducted between the 1960s and 1990s. The landmark Swedish Two-County Trial demonstrated a 31\% reduction in breast cancer mortality among women aged 40--74 who were invited to mammographic screening compared with controls \citep{tabar2003mammography}. Subsequent trials in New York (HIP trial), Canada (CNBSS-1 and CNBSS-2), Edinburgh, Malm\"{o}, Stockholm, and G\"{o}teborg produced varying results, with mortality reductions ranging from null effects to approximately 30\% \citep{nystrom2002long, miller2014twenty}. Meta-analyses of these trials have generally concluded that mammographic screening is associated with a statistically significant reduction in breast cancer mortality on the order of 15--20\% for women aged 50--69, with more uncertain benefits for women aged 40--49 \citep{nelson2009screening, gotzsche2013screening}.

The interpretation of this trial evidence has been contentious. Critics have pointed to methodological limitations in several trials, including suboptimal randomization procedures, contamination between study arms, and the use of outdated imaging technology that may not reflect contemporary mammographic performance \citep{gotzsche2013screening, miller2014twenty}. The Canadian National Breast Screening Study (CNBSS), which uniquely found no mortality benefit from mammographic screening, has been particularly controversial, with debates centering on whether its randomization process was compromised \citep{miller2014twenty, kopans2014cnbss}. Proponents of screening have argued that observational studies conducted in the service screening era, which benefit from modern digital mammography technology and quality assurance programs, provide more relevant evidence of screening effectiveness \citep{broeders2012impact, paci2012summary}.

Against this evidentiary backdrop, major medical organizations have issued screening guidelines that differ substantially in their recommendations. The USPSTF has undergone a notable evolution in its breast cancer screening recommendations. In 2009, the Task Force recommended against routine screening for women aged 40--49, suggesting instead that the decision be individualized based on patient context and values \citep{USPSTF2009breast}. The 2016 update maintained this position, assigning a ``C'' grade to screening for women aged 40--49, indicating that the net benefit was small and that clinicians should consider individual circumstances \citep{siu2016screening}. In a significant policy shift, the 2024 draft recommendation upgraded screening for women aged 40--49 to a ``B'' grade, recommending biennial mammographic screening for all women beginning at age 40 \citep{USPSTF2024breast}. This change was motivated by new evidence on the increasing incidence of breast cancer among women in their 40s and modeling studies suggesting that earlier initiation of screening could prevent additional breast cancer deaths \citep{mandelblatt2016collaborative}.

The American Cancer Society (ACS) has charted a somewhat different course. In 2015, the ACS updated its guidelines to recommend annual mammographic screening for women aged 45--54, with the option to begin annual screening at age 40, and a transition to biennial screening at age 55 \citep{oeffinger2015breast}. This guideline was notable for its introduction of age-stratified recommendations that reflected the changing risk-benefit balance across the lifespan. The ACS designated its recommendation for screening at ages 45--54 as ``strong'' and the recommendations for ages 40--44 and 55 and older as ``qualified,'' reflecting varying levels of evidentiary confidence.

International guidelines exhibit even greater heterogeneity. The United Kingdom's NHS Breast Screening Programme invites women aged 50--70 for mammographic screening every three years, with trials underway to evaluate extending the age range to 47--73 \citep{marmot2013benefits}. The Canadian Task Force on Preventive Health Care recommends against screening for women aged 40--49 who are not at increased risk and recommends screening every two to three years for women aged 50--74 \citep{klarenbach2018recommendations}. The European Commission Initiative on Breast Cancer has recommended biennial mammographic screening for women aged 50--69, with conditional recommendations for extension to ages 45--49 and 70--74 \citep{defined_europerecs2022}. These international variations reflect differing interpretations of the same evidence base, different weighting of benefits versus harms, and different healthcare system contexts.

A central concern in the screening debate is the problem of overdiagnosis---the detection of cancers through screening that would never have caused symptoms or death if left undetected \citep{welch2016breast}. Overdiagnosed cancers include both slow-growing invasive cancers and ductal carcinoma in situ (DCIS), which may never progress to invasive disease \citep{esserman2014overdiagnosis}. Because clinicians cannot currently distinguish overdiagnosed cancers from those that will progress, all screen-detected cancers are typically treated, subjecting some women to the harms of surgery, radiation, chemotherapy, or endocrine therapy for conditions that posed no threat to their health \citep{puliti2012overdiagnosis}. Estimates of the magnitude of overdiagnosis vary enormously---from approximately 1\% to over 50\% of screen-detected cancers---depending on the analytical methodology, comparison group, and time horizon employed \citep{biesheuvel2007policing, etzioni2013overdiagnosis, bleyer2012effect}. This uncertainty about overdiagnosis is a major driver of guideline disagreements and underscores the importance of understanding how women value the trade-offs inherent in screening decisions.

An additional consideration that has gained prominence in recent years is the role of breast density. Women with heterogeneously dense or extremely dense breast tissue face both increased breast cancer risk and reduced mammographic sensitivity, as dense tissue can obscure cancers on mammography \citep{boyd2007mammographic, sprague2014prevalence}. This has led to legislative mandates in many U.S. states requiring that women be notified of their breast density following mammography and has prompted investigation of supplemental screening modalities---including ultrasound, MRI, and contrast-enhanced mammography---for women with dense breasts \citep{berg2012detection, comstock2020comparison}. The integration of breast density into screening recommendations represents an emerging area where patient preferences are particularly relevant, as women must weigh the potential benefits of supplemental screening against additional costs, time, false-positive results, and procedural discomfort.

\subsection{Patient Preferences in Cancer Screening}
\label{subsec:patient_preferences}

The importance of patient preferences in healthcare decision-making has been increasingly recognized over the past two decades, driven by the patient-centered care movement and the broader shift toward shared decision-making \citep{barry2012shared, elwyn2012shared}. Shared decision-making is predicated on the principle that for preference-sensitive decisions---those in which the optimal choice depends on how individuals weigh potential benefits against potential harms---patients should be active participants in the decision-making process \citep{stiggelbout2012shared}. Breast cancer screening is widely recognized as a paradigmatic preference-sensitive decision, particularly for women in age groups where the balance of benefits and harms is most uncertain \citep{hersch2013women}.

Qualitative research has provided valuable insights into the factors that influence women's screening decisions. Studies employing interviews and focus groups have identified several recurring themes, including the importance of test accuracy and the fear of missed cancers, concerns about false-positive results and unnecessary biopsies, the perceived burden of the screening procedure itself (including pain, time, and inconvenience), cost considerations (particularly out-of-pocket expenses), the influence of healthcare provider recommendations, and the role of family history and perceived personal risk \citep{waller2014understanding, whelehan2013review, nelson2020factors}. Cultural factors, health literacy, and prior screening experiences also shape women's attitudes toward and engagement with breast cancer screening \citep{pasick2009behavioral, documet2015perspectives}.

Several barriers to screening uptake have been consistently identified in the literature. Financial barriers, including co-payments, deductibles, and indirect costs such as transportation and time off work, remain significant obstacles to screening participation, particularly among socioeconomically disadvantaged women \citep{trivedi2008effect, sabatino2015effectiveness}. Structural barriers such as geographic access to screening facilities, appointment availability, and lack of a usual source of healthcare also contribute to disparities in screening utilization \citep{ahmed2010disparities}. Psychological barriers, including anxiety about potential results, prior negative screening experiences, and fatalism about cancer outcomes, have been documented across diverse populations \citep{consedine2004fear, jones2014understanding}. Understanding how women weigh these various considerations is essential for designing screening programs that maximize both uptake and alignment with patient values.

The concept of preference-sensitive care has important implications for health policy. If women's preferences systematically diverge from expert-recommended screening strategies, this may indicate a need for either better patient education or more flexible screening programs that accommodate diverse preference profiles \citep{mulley2012patients}. Conversely, if women's preferences largely align with clinical recommendations, this provides additional justification for current guidelines from the perspective of patient welfare. Quantitative evidence on the strength and direction of women's preferences is therefore critical for informing both clinical practice and policy decisions.

\subsection{Discrete Choice Experiments in Health}
\label{subsec:dce_methodology}

Discrete choice experiments represent a rigorous stated preference methodology for eliciting and quantifying individual preferences for goods or services described by multiple attributes \citep{ryan2001using, lancsar2008conducting}. The theoretical foundation of DCEs rests on two complementary frameworks. Lancaster's characteristics theory of value \citep{lancaster1966new} posits that consumers derive utility not from goods per se but from the characteristics or attributes that goods possess. McFadden's random utility theory \citep{mcfadden1974conditional} provides the econometric framework for analyzing discrete choice data, modeling the probability that an individual selects a particular alternative as a function of its observed attributes and an unobserved random component.

The DCE methodology involves presenting respondents with a series of choice tasks, each containing two or more hypothetical alternatives (profiles) described by systematically varied attribute levels, and asking respondents to select their most preferred alternative \citep{louviere2000stated, hensher2015applied}. The attribute levels are varied according to a statistical experimental design that ensures sufficient variation for parameter estimation while maintaining orthogonality or near-orthogonality among attributes \citep{street2007construction, rose2009constructing}. By analyzing the pattern of choices across multiple tasks, researchers can estimate the relative importance of each attribute, calculate marginal rates of substitution between attributes, and derive willingness-to-pay values when a monetary attribute is included \citep{train2009discrete}.

DCEs offer several advantages over alternative preference elicitation methods in health economics. Compared with contingent valuation, which directly asks respondents for their WTP for a defined good, DCEs provide richer information about the trade-offs individuals make among multiple attributes simultaneously and are generally considered less susceptible to strategic bias \citep{hanley2001choice, ryan2004using}. Unlike health state valuation methods such as the time trade-off (TTO) or standard gamble (SG), which focus on valuing health outcomes, DCEs can incorporate process attributes (e.g., convenience, waiting time) and non-health attributes (e.g., cost, provider type) that may be important determinants of healthcare choices \citep{ryan2008using}. Furthermore, DCEs allow for the estimation of preference heterogeneity through advanced econometric models, providing insights into how preferences vary across population subgroups \citep{greene2003latent, hole2008modelling}.

The application of DCEs in health economics has grown exponentially since the early 2000s. A systematic review by \citet{debbekkergrob2012dcereview} identified 114 health-related DCE studies published between 2001 and 2008, compared with only 34 published before 2001. A subsequent review by \citet{clark2014discrete} documented continued rapid growth, with DCEs applied across a wide range of health topics including treatment preferences, health service delivery, workforce planning, and health technology assessment. More recent reviews have noted increasing methodological sophistication in health-related DCEs, including greater use of efficient experimental designs, more complex model specifications, and attention to issues such as attribute non-attendance and response quality \citep{soekhai2019discrete, debekerrgrob2015sample}.

Several methodological considerations are critical to the validity and reliability of DCE results. Attribute selection should be guided by a combination of literature review, qualitative research with the target population, and expert consultation to ensure that the most relevant attributes are included and that attribute levels span a realistic and policy-relevant range \citep{coastdce2012, klojgaard2012designing}. The experimental design should balance statistical efficiency with cognitive feasibility, as overly complex choice tasks may lead to respondent fatigue, simplifying heuristics, or random responding \citep{johnson2013constructing, bech2011exploring}. Sample size calculations for DCEs should account for the number of parameters to be estimated, the number of choice tasks per respondent, and the expected model specification \citep{debekerrgrob2015sample, orme2010getting}. Finally, the econometric analysis should employ model specifications appropriate to the research questions, with increasing recognition that models accommodating preference heterogeneity (e.g., mixed logit, latent class) generally outperform the basic conditional logit model both in terms of model fit and substantive insight \citep{train2009discrete, hensher2015applied}.

\subsection{Previous DCE Studies on Breast Cancer Screening}
\label{subsec:previous_dces}

A growing body of literature has applied the DCE methodology to investigate preferences for breast cancer screening. Table~\ref{tab:previous_dce_studies} summarizes the key characteristics and findings of selected previous studies.

\begin{landscape}
\begin{table}[htbp]
\centering
\small
\caption{Summary of Previous DCE Studies on Breast Cancer Screening}
\label{tab:previous_dce_studies}
\begin{threeparttable}
\begin{tabularx}{\linewidth}{>{\raggedright\arraybackslash}p{2.5cm} >{\raggedright\arraybackslash}p{1.5cm} >{\raggedright\arraybackslash}p{1.5cm} >{\raggedright\arraybackslash}p{4cm} >{\raggedright\arraybackslash}p{2.5cm} >{\raggedright\arraybackslash}p{5cm} >{\raggedright\arraybackslash}p{2.5cm}}
\toprule
Study & Country & Sample Size & Attributes & Model & Key Findings & Limitations \\
\midrule
\citet{marshall2010conjoint} & Canada & 719 & Mortality reduction, overdiagnosis, false positives, screening frequency & CL & Mortality reduction most important; WTP for screening programs varied by age & Limited attribute set \\
\addlinespace
\citet{de2012labels} & Netherlands & 502 & Test sensitivity, test specificity, frequency, pain level & CL, MXL & Sensitivity valued most; label effects present for screening tests & No cost attribute \\
\addlinespace
\citet{van2014preferences} & Netherlands & 896 & Risk reduction, false positives, overdiagnosis rate, frequency & CL & Risk reduction dominated; overdiagnosis influenced choices less than expected & No process attributes \\
\addlinespace
\citet{muhlbacher2016preferences} & Germany & 658 & Accuracy, frequency, cost, waiting time, radiation exposure & CL, LC & Accuracy and cost most important; 2 latent classes identified & Online sample only \\
\addlinespace
\citet{gyrd2016discrete} & Denmark & 612 & Sensitivity, specificity, screening interval, procedure type & MXL & Preference heterogeneity for interval; sensitivity valued highly & Small sample \\
\addlinespace
\citet{mansfield2016stated} & Australia & 847 & Screening method, accuracy, cost, frequency, time & CL & Strong preference for familiar modalities; cost sensitivity varied by income & No false-positive attribute \\
\addlinespace
\citet{phillips2019preferences} & UK & 1,004 & Mortality reduction, overdiagnosis, false positives, test type & CL, LC & 3 classes: benefit-focused, harm-averse, indifferent; age was key predictor & Hypothetical framing \\
\addlinespace
\citet{deverill2021preferences} & UK & 534 & Screening method, sensitivity, frequency, travel time, cost & CL & Travel time and cost were barriers; preference for NHS-provided services & UK-specific context \\
\addlinespace
\citet{wortley2020breast} & Australia & 1,216 & Screening method, accuracy, false-positive rate, interval, out-of-pocket cost & CL, MXL & Cost and false-positive rate most important; preference heterogeneity observed & No discomfort attribute \\
\addlinespace
\citet{petrova2020dce} & Europe & 2,431 & Detection rate, overdiagnosis, false positives, screening interval & CL & Detection rate most valued; cross-country variation in trade-offs & No cost attribute; hypothetical \\
\bottomrule
\end{tabularx}
\begin{tablenotes}
\small
\item \textit{Notes:} CL = conditional logit; MXL = mixed logit; LC = latent class model. Studies are listed in approximate chronological order. This table includes selected studies and is not exhaustive of the full literature.
\end{tablenotes}
\end{threeparttable}
\end{table}
\end{landscape}

Several patterns emerge from the existing literature. First, test accuracy---whether measured as sensitivity, detection rate, or mortality reduction---has consistently been identified as one of the most important attributes influencing women's screening preferences \citep{marshall2010conjoint, de2012labels, gyrd2016discrete}. Women place substantial value on screening tests that offer higher probabilities of detecting cancer when present, which aligns with the intuitive importance of the screening test's primary function. Second, cost has emerged as a significant determinant of preferences in studies that include a monetary attribute, with women generally exhibiting negative preferences for higher out-of-pocket costs, though the magnitude of cost sensitivity varies across income levels and healthcare system contexts \citep{muhlbacher2016preferences, mansfield2016stated, wortley2020breast}.

Third, the evidence on screening frequency preferences is mixed. While some studies have found that women prefer more frequent screening intervals \citep{marshall2010conjoint}, others have documented heterogeneity in frequency preferences, with some women preferring less frequent screening, potentially reflecting an awareness of the cumulative risks associated with repeated screening \citep{gyrd2016discrete, van2014preferences}. Fourth, false-positive rates and overdiagnosis have been found to influence preferences, though generally to a lesser extent than detection rates and cost \citep{van2014preferences, phillips2019preferences, wortley2020breast}. This asymmetry may reflect the inherent difficulty individuals face in comprehending and evaluating probabilistic harm information \citep{gigerenzer2007helping}.

Studies employing latent class analysis have consistently found evidence of preference heterogeneity, with distinct subgroups of women exhibiting different preference patterns \citep{phillips2019preferences, muhlbacher2016preferences}. \citet{phillips2019preferences} identified three latent classes in a UK sample: benefit-focused women who prioritized mortality reduction, harm-averse women who were most concerned about overdiagnosis and false positives, and a relatively indifferent group with weak preferences across all attributes. Age, education, prior screening experience, and family history of breast cancer have been identified as predictors of class membership, suggesting that preference heterogeneity is systematically related to observable characteristics \citep{phillips2019preferences, muhlbacher2016preferences}.

\subsection{Research Gap and Contribution}
\label{subsec:research_gap}

Despite the valuable contributions of previous DCE studies on breast cancer screening, several important gaps remain in the literature. First, few studies have included a comprehensive set of attributes that spans both clinical attributes (e.g., sensitivity, false-positive rate) and process attributes (e.g., physical discomfort, waiting time for results). Qualitative research consistently identifies procedural comfort and convenience as important factors influencing women's screening decisions \citep{whelehan2013review, nelson2020factors}, yet these attributes have been underrepresented in the DCE literature. Second, the majority of previous studies have relied on conditional logit models, which impose restrictive assumptions including taste homogeneity and the independence of irrelevant alternatives \citep{hausman1984specification}. While several recent studies have employed mixed logit or latent class models \citep{gyrd2016discrete, phillips2019preferences, wortley2020breast}, there is a need for more studies that systematically compare results across model specifications to assess the robustness of findings and the extent of preference heterogeneity.

Third, the evolving guideline landscape---particularly the USPSTF's 2024 recommendation to lower the screening starting age to 40---motivates renewed investigation of women's preferences in a contemporary context. Women's preferences may have shifted in response to public discourse about screening benefits and harms, changing healthcare costs, or the availability of new screening technologies. Fourth, few studies have calculated WTP estimates for a comprehensive set of screening attributes, limiting the ability of policymakers to evaluate the economic value women place on screening program features. WTP estimates are particularly relevant for coverage decisions, cost-effectiveness analyses, and the design of benefit packages.

The present study addresses these gaps by conducting a DCE that incorporates seven attributes spanning both clinical and process dimensions of breast cancer screening. We employ three complementary econometric models---conditional logit, mixed logit, and latent class---to provide a comprehensive characterization of women's preferences and preference heterogeneity. We calculate WTP estimates for all non-cost attributes, providing policy-relevant valuations that can inform screening program design and coverage decisions. To our knowledge, this is among the first DCE studies on breast cancer screening to simultaneously incorporate physical discomfort and waiting time for results alongside traditional clinical and cost attributes, and to employ a full suite of advanced discrete choice models in the analysis.


% Methods
\section{Methods}
\label{sec:methods}

\subsection{Theoretical Framework}
\label{subsec:theoretical_framework}

This study is grounded in random utility theory (RUT), which provides the theoretical basis for analyzing discrete choice data \citep{mcfadden1974conditional, train2009discrete}. RUT posits that individuals are rational utility maximizers who, when faced with a choice among a set of mutually exclusive alternatives, select the alternative that yields the highest utility. Following Lancaster's characteristics theory of value \citep{lancaster1966new}, we assume that the utility derived from a breast cancer screening program is a function of its constituent attributes rather than the program as a whole.

Formally, the utility that individual $n$ derives from choosing alternative $i$ in choice situation $j$ can be decomposed into a systematic (observable) component $V_{nij}$ and a random (unobservable) component $\varepsilon_{nij}$:

\begin{equation}
\label{eq:utility}
U_{nij} = V_{nij} + \varepsilon_{nij}
\end{equation}

\noindent where the systematic component is specified as a linear-in-parameters function of the attribute levels:

\begin{equation}
\label{eq:systematic_utility}
V_{nij} = \boldsymbol{\beta}' \mathbf{x}_{nij}
\end{equation}

\noindent where $\mathbf{x}_{nij}$ is a vector of attribute levels characterizing alternative $i$ in choice situation $j$ as faced by individual $n$, and $\boldsymbol{\beta}$ is a vector of preference parameters (utility weights) to be estimated. Individual $n$ will choose alternative $i$ over alternative $k$ in choice situation $j$ if and only if $U_{nij} > U_{nkj}$ for all $k \neq i$ in the choice set $C_{nj}$.

The distributional assumption imposed on the random component $\varepsilon_{nij}$ determines the choice model. When $\varepsilon_{nij}$ is assumed to be independently and identically distributed (IID) according to a Type I extreme value (Gumbel) distribution, the conditional logit (CL) model results \citep{mcfadden1974conditional}. Relaxing the IID assumption leads to more flexible model specifications, including the mixed logit and latent class models described in Section~\ref{subsec:econometric_analysis}.

\subsection{Study Design}
\label{subsec:study_design}

We employed a labeled DCE in which respondents were asked to choose among hypothetical breast cancer screening programs described by seven attributes. The attributes and their levels were selected through a systematic process involving three stages: (1) a comprehensive review of the clinical and DCE literature on breast cancer screening (Section~\ref{sec:literature_review}); (2) consultation with a multidisciplinary expert panel comprising breast radiologists, oncologists, health economists, and patient advocates; and (3) a qualitative pilot study consisting of semi-structured interviews and cognitive debriefing with 20 women from the target population. This process ensured that the final attribute set captured the dimensions of breast cancer screening most relevant to women's decision-making while remaining cognitively manageable within a choice experiment framework.

Table~\ref{tab:attributes_levels} presents the seven attributes, their levels, and the rationale for their inclusion. Categorical attributes (screening method, screening frequency, waiting time for results, and physical discomfort) were effects-coded, while continuous attributes (out-of-pocket cost, sensitivity, and false-positive rate) were entered as continuous variables in the utility function.

\begin{table}[htbp]
\centering
\caption{Attributes and Levels Used in the Discrete Choice Experiment}
\label{tab:attributes_levels}
\begin{threeparttable}
\begin{tabularx}{\textwidth}{>{\raggedright\arraybackslash}p{3cm} >{\raggedright\arraybackslash}p{4.5cm} >{\raggedright\arraybackslash}X}
\toprule
Attribute & Levels & Rationale \\
\midrule
Screening method & Mammography\textsuperscript{a}; MRI; Ultrasound & Reflects current and emerging modalities \citep{lord2007systematic, berg2012detection} \\
\addlinespace
Screening frequency & Annual\textsuperscript{a}; Every 2 years; Every 3 years & Spans range of guideline recommendations \citep{USPSTF2024breast, oeffinger2015breast} \\
\addlinespace
Out-of-pocket cost & \$0; \$50; \$150; \$300 & Reflects variation in copayments and uninsured costs \citep{trivedi2008effect} \\
\addlinespace
Sensitivity (detection rate) & 70\%; 85\%; 95\% & Spans observed sensitivity range across modalities \citep{nelson2009screening, lord2007systematic} \\
\addlinespace
False-positive rate & 5\%; 10\%; 15\% & Reflects range of observed false-positive rates \citep{hubbard2011cumulative} \\
\addlinespace
Waiting time for results & Same day\textsuperscript{a}; 1 week; 3 weeks & Based on qualitative findings on result anxiety \citep{whelehan2013review} \\
\addlinespace
Physical discomfort & None\textsuperscript{a}; Mild; Moderate & Reflects patient-reported screening experiences \citep{nelson2020factors, whelehan2013review} \\
\bottomrule
\end{tabularx}
\begin{tablenotes}
\small
\item \textsuperscript{a} Reference level for effects coding. For continuous attributes (cost, sensitivity, false-positive rate), the variable is entered linearly.
\end{tablenotes}
\end{threeparttable}
\end{table}

\subsection{Experimental Design}
\label{subsec:experimental_design}

The full factorial design for the seven attributes with the specified levels comprises $3 \times 3 \times 4 \times 3 \times 3 \times 3 \times 3 = 2{,}916$ possible screening profiles. As presenting all possible pairwise combinations to respondents is infeasible, we employed a D-optimal fractional factorial design to generate a manageable subset of choice tasks that maximizes the statistical efficiency of parameter estimates \citep{street2007construction, rose2009constructing}.

The experimental design was generated using Ngene software (ChoiceMetrics, version 1.3), which employs a modified Fedorov algorithm to identify the design with the minimum D-error \citep{choicemetrics2018ngene}. The design comprised 12 choice tasks, each containing two experimentally designed screening program alternatives and a constant opt-out alternative (``I would choose not to be screened''). The inclusion of an opt-out alternative reflects the reality that women can decline screening and allows for estimation of participation probabilities \citep{lancsar2007estimating, campbell2018including}.

To reduce respondent burden, the 12 choice tasks were blocked into two versions of six choice tasks each, with respondents randomly assigned to one version \citep{louviere2000stated}. The blocking was performed using a design criterion that maintained D-optimality within each block. The final design achieved a D-error of 0.0023, indicating high statistical efficiency. An example choice task is presented in Figure~\ref{fig:example_choice_task}.

\begin{figure}[htbp]
\centering
\fbox{
\begin{minipage}{0.9\textwidth}
\vspace{0.5em}
\textbf{Choice Task Example}\\[0.5em]
Imagine you are deciding which breast cancer screening program to participate in. Please compare the two screening programs described below and indicate which you would prefer. You may also choose not to participate in any screening.\\[1em]
\begin{tabular}{lccc}
\toprule
& \textbf{Program A} & \textbf{Program B} & \textbf{No Screening} \\
\midrule
Screening method & MRI & Mammography & --- \\
Frequency & Every 2 years & Annual & --- \\
Out-of-pocket cost & \$150 & \$50 & \$0 \\
Detection rate & 95\% & 85\% & --- \\
False-positive rate & 10\% & 15\% & --- \\
Waiting time & 1 week & Same day & --- \\
Physical discomfort & None & Moderate & --- \\
\midrule
\textbf{I would choose:} & $\square$ & $\square$ & $\square$ \\
\bottomrule
\end{tabular}
\vspace{0.5em}
\end{minipage}
}
\caption{Example choice task from the discrete choice experiment.}
\label{fig:example_choice_task}
\end{figure}

\subsection{Sample and Recruitment}
\label{subsec:sample_recruitment}

The target population for this study comprised women aged 40--74 residing in the United States. This age range was selected to align with the age groups for which breast cancer screening is most actively recommended and debated \citep{USPSTF2024breast, oeffinger2015breast}. Inclusion criteria were: (1) female sex; (2) age 40--74 years; (3) no prior diagnosis of breast cancer; (4) ability to read and understand English; and (5) provision of informed consent. Women with a prior breast cancer diagnosis were excluded because their screening preferences may be systematically different from those of the general screening-eligible population and because their choices would reflect treatment-influenced perceptions rather than primary screening considerations.

The minimum sample size was calculated using the formula proposed by \citet{debekerrgrob2015sample}:

\begin{equation}
\label{eq:sample_size}
N \geq \frac{500c}{t \times a}
\end{equation}

\noindent where $N$ is the minimum number of respondents, $c$ is the largest number of levels for any single attribute (4, for out-of-pocket cost), $t$ is the number of choice tasks per respondent (6), and $a$ is the number of alternatives per choice task excluding the opt-out (2). This yields a minimum sample size of $N \geq 500 \times 4 / (6 \times 2) = 167$. To accommodate subgroup analyses, latent class modeling, and potential data quality exclusions, we targeted a sample of 500 completed responses, substantially exceeding the minimum requirement. This target is consistent with sample sizes recommended for mixed logit estimation with multiple random parameters \citep{train2009discrete, orme2010getting}.

Participants were recruited through two channels: (1) an online survey panel (Prolific Academic) and (2) clinic-based recruitment at three breast imaging centers. The dual recruitment strategy was designed to enhance sample diversity and to allow for comparison of preferences between online and clinic-based participants. Online panel participants were recruited using Prolific's demographic pre-screening filters to match the target population. Clinic-based participants were recruited by research assistants who approached women in the waiting areas of participating breast imaging centers.

\subsection{Survey Administration}
\label{subsec:survey_administration}

The survey was programmed and administered using Qualtrics (Qualtrics, Provo, UT). The survey flow was structured as follows:

\begin{enumerate}
\item \textbf{Information and consent:} Participants received a study information sheet describing the purpose, procedures, risks, and benefits of participation, followed by an electronic informed consent form.

\item \textbf{Eligibility screening:} A brief set of questions confirmed that participants met the inclusion criteria.

\item \textbf{DCE tutorial:} Participants received an interactive tutorial explaining the choice task format, the attributes and their levels (with definitions and visual aids), and the concept of trading off between attributes. The tutorial was developed following best practices for DCE administration \citep{ryan2001using, coastdce2012}.

\item \textbf{Practice choice task:} Participants completed one practice choice task with feedback to ensure comprehension before proceeding to the experimental tasks.

\item \textbf{Choice tasks:} Participants completed six randomized choice tasks from their assigned design block.

\item \textbf{Follow-up questions:} After completing the choice tasks, participants answered questions about task difficulty, decision strategies, and attribute non-attendance to support data quality assessment.

\item \textbf{Sociodemographic questionnaire:} Participants provided information on age, race/ethnicity, education, household income, health insurance status, prior screening history, family history of breast cancer, and perceived breast cancer risk.

\item \textbf{Debriefing:} Participants received a debriefing statement explaining the study's purpose and providing links to breast cancer screening resources.
\end{enumerate}

Several data quality measures were implemented. These included an attention check question embedded in the sociodemographic section, a dominance test choice task (in which one alternative was clearly superior on all attributes) to identify respondents who were not engaging meaningfully with the tasks, and minimum completion time thresholds to screen out ``speeders'' \citep{johnson2013constructing}. Respondents who failed the attention check, selected the dominated alternative in the dominance test, or completed the survey in less than one-third of the median completion time were flagged for exclusion from the primary analysis.

\subsection{Econometric Analysis}
\label{subsec:econometric_analysis}

Choice data were analyzed using three complementary econometric models of increasing flexibility: the conditional logit (CL) model, the mixed logit (MXL) model, and the latent class (LC) model. All models were estimated using maximum likelihood or simulated maximum likelihood methods in R (version 4.3.0) using the \texttt{mlogit} \citep{croissant2020mlogit} and \texttt{gmnl} \citep{sarrias2017gmnl} packages. Stata 18 \citep{statacorp2023stata} was used for robustness checks.

\subsubsection{Conditional Logit Model}

The CL model, also known as the multinomial logit model, assumes that the unobserved components of utility ($\varepsilon_{nij}$) are IID Type I extreme value distributed \citep{mcfadden1974conditional}. Under this assumption, the probability that individual $n$ chooses alternative $i$ from choice set $C_{nj}$ is given by:

\begin{equation}
\label{eq:cl_probability}
P_{n}(i \mid C_{nj}) = \frac{\exp(V_{nij})}{\sum_{k \in C_{nj}} \exp(V_{nkj})}
\end{equation}

The systematic utility function for each screening alternative was specified as:

\begin{equation}
\label{eq:utility_specification}
\begin{aligned}
V_{nij} = \; & \beta_1 \cdot \text{Method}_{\text{MRI}} + \beta_2 \cdot \text{Method}_{\text{US}} + \beta_3 \cdot \text{Freq}_{\text{biennial}} + \beta_4 \cdot \text{Freq}_{\text{triennial}} \\
& + \beta_5 \cdot \text{Cost} + \beta_6 \cdot \text{Sensitivity} + \beta_7 \cdot \text{FPR} \\
& + \beta_8 \cdot \text{Wait}_{\text{1 week}} + \beta_9 \cdot \text{Wait}_{\text{3 weeks}} \\
& + \beta_{10} \cdot \text{Pain}_{\text{mild}} + \beta_{11} \cdot \text{Pain}_{\text{moderate}} + \beta_{\text{opt-out}} \cdot \text{OptOut}
\end{aligned}
\end{equation}

\noindent where the reference levels for effects-coded attributes are mammography (method), annual (frequency), same day (waiting time), and none (physical discomfort). The cost attribute was entered in units of \$100, the sensitivity attribute in units of 10 percentage points, and the false-positive rate in units of 5 percentage points to facilitate interpretation. The opt-out alternative was captured by an alternative-specific constant ($\beta_{\text{opt-out}}$) that takes the value 1 for the ``no screening'' alternative and 0 otherwise.

The CL model was estimated by maximizing the log-likelihood function:

\begin{equation}
\label{eq:cl_loglikelihood}
\mathcal{LL}(\boldsymbol{\beta}) = \sum_{n=1}^{N} \sum_{j=1}^{J} \sum_{i \in C_{nj}} y_{nij} \ln P_{n}(i \mid C_{nj})
\end{equation}

\noindent where $y_{nij} = 1$ if individual $n$ chose alternative $i$ in choice situation $j$, and $y_{nij} = 0$ otherwise.

While the CL model provides consistent and efficient estimates under its maintained assumptions, it imposes two restrictive properties: taste homogeneity (all individuals share the same preference parameters $\boldsymbol{\beta}$) and the independence of irrelevant alternatives (IIA), which requires that the ratio of choice probabilities for any two alternatives be independent of the attributes of other alternatives in the choice set \citep{hausman1984specification, train2009discrete}. These restrictions motivate the estimation of more flexible models.

\subsubsection{Mixed Logit Model}

The MXL model, also known as the random parameters logit model, relaxes both the taste homogeneity and IIA assumptions by allowing preference parameters to vary randomly across individuals \citep{train2009discrete, hensher2015applied}. Each individual $n$ is assumed to have a unique vector of preference parameters $\boldsymbol{\beta}_n$ drawn from a continuous mixing distribution $f(\boldsymbol{\beta} \mid \boldsymbol{\theta})$, where $\boldsymbol{\theta}$ denotes the distributional parameters (e.g., mean and variance for normally distributed parameters).

The unconditional choice probability in the MXL model is obtained by integrating the CL probability over the distribution of $\boldsymbol{\beta}$:

\begin{equation}
\label{eq:mxl_probability}
P_{n}(i \mid C_{nj}) = \int \frac{\exp(\boldsymbol{\beta}_n' \mathbf{x}_{nij})}{\sum_{k \in C_{nj}} \exp(\boldsymbol{\beta}_n' \mathbf{x}_{nkj})} f(\boldsymbol{\beta} \mid \boldsymbol{\theta}) \, d\boldsymbol{\beta}
\end{equation}

This integral does not have a closed-form solution and is approximated using simulated maximum likelihood estimation with $R = 1{,}000$ Halton draws per individual to ensure stable estimation \citep{train2009discrete, bhat2001quasi}. The simulated log-likelihood function is:

\begin{equation}
\label{eq:mxl_simulated_ll}
\mathcal{SLL}(\boldsymbol{\theta}) = \sum_{n=1}^{N} \ln \left[ \frac{1}{R} \sum_{r=1}^{R} \prod_{j=1}^{J} \frac{\exp(\boldsymbol{\beta}_n^{(r)\prime} \mathbf{x}_{nij})}{\sum_{k \in C_{nj}} \exp(\boldsymbol{\beta}_n^{(r)\prime} \mathbf{x}_{nkj})} \right]
\end{equation}

All non-cost parameters were specified as random with normal distributions to allow for the possibility that preferences for screening attributes may be both positive and negative across the population. The cost parameter was held fixed (non-random) to ensure that the distribution of WTP estimates is well-defined and bounded, following the recommendation of \citet{hole2008modelling} and \citet{train2009discrete}. Specifically, when the cost coefficient is fixed, the WTP for each attribute is distributed normally if the attribute coefficient is normally distributed, facilitating straightforward interpretation.

We tested alternative distributional assumptions including lognormal (for parameters expected to have a definite sign, such as sensitivity) and triangular distributions, with model selection guided by the Bayesian Information Criterion (BIC) and substantive interpretability.

\subsubsection{Latent Class Model}

The LC model provides an alternative approach to capturing preference heterogeneity by assuming that the population consists of a finite number of $S$ latent classes (segments), each characterized by a distinct set of preference parameters \citep{greene2003latent, boxall2002latent}. Unlike the MXL model, which assumes continuously distributed heterogeneity, the LC model posits discrete heterogeneity, allowing for the identification of distinct preference profiles that may correspond to meaningful population subgroups.

The probability that individual $n$ chooses alternative $i$ in the LC model is a weighted average of class-specific CL probabilities:

\begin{equation}
\label{eq:lc_probability}
P_{n}(i \mid C_{nj}) = \sum_{s=1}^{S} \pi_{ns} \cdot \frac{\exp(\boldsymbol{\beta}_s' \mathbf{x}_{nij})}{\sum_{k \in C_{nj}} \exp(\boldsymbol{\beta}_s' \mathbf{x}_{nkj})}
\end{equation}

\noindent where $\boldsymbol{\beta}_s$ is the vector of preference parameters for class $s$, and $\pi_{ns}$ is the probability that individual $n$ belongs to class $s$, which sums to unity across classes. Class membership probabilities are modeled as a function of individual characteristics using a multinomial logit specification:

\begin{equation}
\label{eq:class_membership}
\pi_{ns} = \frac{\exp(\boldsymbol{\gamma}_s' \mathbf{z}_n)}{\sum_{r=1}^{S} \exp(\boldsymbol{\gamma}_r' \mathbf{z}_n)}
\end{equation}

\noindent where $\mathbf{z}_n$ is a vector of individual-specific covariates (e.g., age, income, education, screening history) and $\boldsymbol{\gamma}_s$ is the corresponding vector of class membership parameters, with $\boldsymbol{\gamma}_S = \mathbf{0}$ for identification.

The optimal number of latent classes was determined by estimating models with $S = 2, 3, 4, 5$ classes and comparing model fit using the BIC:

\begin{equation}
\label{eq:bic}
\text{BIC} = -2 \mathcal{LL} + K \ln(N)
\end{equation}

\noindent where $\mathcal{LL}$ is the maximized log-likelihood, $K$ is the number of estimated parameters, and $N$ is the number of respondents. The model with the lowest BIC was selected as the preferred specification, with consideration also given to the Akaike Information Criterion (AIC), the consistent AIC (CAIC), and substantive interpretability of the resulting classes \citep{nylund2007deciding}.

\subsubsection{Willingness-to-Pay Estimation}

WTP estimates for each non-cost attribute were calculated as the negative ratio of the attribute coefficient to the cost coefficient:

\begin{equation}
\label{eq:wtp}
\text{WTP}_{\text{attribute}} = -\frac{\beta_{\text{attribute}}}{\beta_{\text{cost}}}
\end{equation}

This ratio represents the marginal rate of substitution between the attribute and cost, interpreted as the maximum amount an individual would be willing to pay for a one-unit improvement in the attribute (or for a move from the reference level to the specified level for effects-coded attributes), all else equal.

Confidence intervals for WTP estimates were calculated using the Krinsky-Robb parametric bootstrapping method \citep{krinsky1986approximating}, which involves drawing 10,000 replications from the asymptotic multivariate normal distribution of the estimated parameters and computing the WTP ratio for each draw. The 2.5th and 97.5th percentiles of the resulting WTP distribution were used to construct 95\% confidence intervals. We also computed WTP confidence intervals using the delta method for comparison \citep{hole2007fitting}.

In the MXL model, where attribute coefficients are randomly distributed, the WTP for each attribute follows a derived distribution. Because the cost coefficient was held fixed, the WTP distribution mirrors the distribution of the attribute coefficient. We report the mean WTP and the standard deviation of the WTP distribution.

In the LC model, class-specific WTP estimates were calculated by applying Equation~\eqref{eq:wtp} separately within each class, yielding a set of WTP values that characterize the distinct preference profiles of each latent segment.

\subsection{Ethical Considerations}
\label{subsec:ethics}

This study was approved by the Institutional Review Board (IRB) of [University Name] (Protocol No. [to be inserted]). All participants provided written informed consent prior to participation. The study was conducted in accordance with the principles of the Declaration of Helsinki and applicable data protection regulations. All survey data were collected anonymously, with no personally identifiable information linked to choice responses. Participants were informed that their participation was voluntary and that they could withdraw at any time without penalty. Online participants recruited through Prolific were compensated at a rate of \$10.00 per hour for their time, consistent with the platform's fair payment guidelines. Clinic-based participants received a \$15 gift card upon completion of the survey.


% Results
\section{Results}
\label{sec:results}

\subsection{Response Rate and Sample Characteristics}
\label{subsec:sample_characteristics}

A total of 714 women accessed the survey, of whom 538 met the eligibility criteria and provided informed consent. After excluding 38 respondents who failed data quality checks (15 failed the attention check, 12 selected the dominated alternative in the dominance test, and 11 completed the survey below the minimum time threshold), the final analytical sample comprised 500 respondents, yielding an effective response rate of 70.0\%.

Table~\ref{tab:demographics} presents the sociodemographic characteristics of the study sample. The mean age of respondents was 54.3 years (SD = 9.7), with approximately equal representation across the 40--49 (31.2\%), 50--59 (35.4\%), and 60--74 (33.4\%) age groups. The majority of respondents identified as non-Hispanic White (62.8\%), followed by Black or African American (15.6\%), Hispanic or Latina (12.4\%), and Asian (6.2\%). With respect to educational attainment, 38.2\% held a bachelor's degree or higher, 29.4\% had completed some college, and 32.4\% had a high school diploma or less. Annual household income was distributed broadly, with 24.0\% reporting less than \$35,000, 38.6\% reporting \$35,000--\$74,999, and 37.4\% reporting \$75,000 or more. The vast majority of respondents (87.2\%) had health insurance. Nearly three-quarters (73.8\%) reported having undergone at least one mammogram previously, and 18.4\% reported a first-degree family history of breast cancer.

\begin{table}[htbp]
\centering
\caption{Sociodemographic Characteristics of the Study Sample (N = 500)}
\label{tab:demographics}
\begin{threeparttable}
\begin{tabular}{lcc}
\toprule
Characteristic & $n$ & \% \\
\midrule
\textbf{Age group} & & \\
\quad 40--49 years & 156 & 31.2 \\
\quad 50--59 years & 177 & 35.4 \\
\quad 60--74 years & 167 & 33.4 \\
\addlinespace
\textbf{Race/Ethnicity} & & \\
\quad Non-Hispanic White & 314 & 62.8 \\
\quad Black or African American & 78 & 15.6 \\
\quad Hispanic or Latina & 62 & 12.4 \\
\quad Asian & 31 & 6.2 \\
\quad Other/Multiracial & 15 & 3.0 \\
\addlinespace
\textbf{Education} & & \\
\quad High school diploma or less & 162 & 32.4 \\
\quad Some college or associate's degree & 147 & 29.4 \\
\quad Bachelor's degree or higher & 191 & 38.2 \\
\addlinespace
\textbf{Annual household income} & & \\
\quad Less than \$35,000 & 120 & 24.0 \\
\quad \$35,000--\$74,999 & 193 & 38.6 \\
\quad \$75,000 or more & 187 & 37.4 \\
\addlinespace
\textbf{Health insurance status} & & \\
\quad Insured & 436 & 87.2 \\
\quad Uninsured & 64 & 12.8 \\
\addlinespace
\textbf{Prior mammography experience} & & \\
\quad Yes & 369 & 73.8 \\
\quad No & 131 & 26.2 \\
\addlinespace
\textbf{Family history of breast cancer} & & \\
\quad Yes (first-degree relative) & 92 & 18.4 \\
\quad No & 408 & 81.6 \\
\bottomrule
\end{tabular}
\begin{tablenotes}
\small
\item \textit{Notes:} Percentages may not sum to 100 due to rounding.
\end{tablenotes}
\end{threeparttable}
\end{table}

\subsection{Conditional Logit Model Results}
\label{subsec:cl_results}

Table~\ref{tab:cl_results} presents the parameter estimates from the CL model. All attribute coefficients were statistically significant at the 5\% level, confirming that each of the seven screening attributes influenced women's choices. The model fit statistics indicated a McFadden pseudo-$R^2$ of 0.247, suggesting a good fit to the data for a discrete choice model \citep{louviere2000stated}.

The cost coefficient was negative and highly significant ($\beta_{\text{cost}} = -0.583$, $p < 0.001$), confirming that women preferred lower out-of-pocket costs, as expected. The sensitivity coefficient was positive and significant ($\beta_{\text{sensitivity}} = 0.842$, $p < 0.001$), indicating a strong preference for higher detection rates. The false-positive rate coefficient was negative and significant ($\beta_{\text{FPR}} = -0.376$, $p < 0.001$), reflecting preferences for lower false-positive rates.

Regarding screening method, MRI received a positive and significant coefficient relative to mammography ($\beta_{\text{MRI}} = 1.048$, $p < 0.001$), while ultrasound received a negative but marginally significant coefficient ($\beta_{\text{US}} = -0.215$, $p = 0.037$). These results suggest that, holding other attributes constant, women preferred MRI over mammography and mammography over ultrasound.

Screening frequency coefficients indicated that women preferred more frequent screening. Biennial screening was associated with a negative utility relative to annual screening ($\beta_{\text{biennial}} = -0.412$, $p < 0.001$), and triennial screening was associated with an even more negative utility ($\beta_{\text{triennial}} = -0.687$, $p < 0.001$), demonstrating a monotonic preference for more frequent screening.

Longer waiting times for results were associated with reduced utility. A one-week wait was associated with a significant negative coefficient relative to same-day results ($\beta_{\text{1 week}} = -0.324$, $p < 0.001$), and a three-week wait was associated with a larger negative coefficient ($\beta_{\text{3 weeks}} = -0.618$, $p < 0.001$). Similarly, increasing levels of physical discomfort were associated with decreasing utility, with mild discomfort ($\beta_{\text{mild}} = -0.198$, $p = 0.004$) and moderate discomfort ($\beta_{\text{moderate}} = -0.456$, $p < 0.001$) both significant relative to no discomfort.

The opt-out constant was negative and significant ($\beta_{\text{opt-out}} = -1.342$, $p < 0.001$), indicating a general preference for participating in some form of screening over opting out entirely. The opt-out alternative was chosen in 11.3\% of choice tasks.

\begin{table}[htbp]
\centering
\caption{Conditional Logit Model Estimates}
\label{tab:cl_results}
\begin{threeparttable}
\begin{tabular}{lccc}
\toprule
Attribute & Coefficient & SE & $p$-value \\
\midrule
\textbf{Screening method} (ref: Mammography) & & & \\
\quad MRI & 1.048 & 0.142 & $<$0.001 \\
\quad Ultrasound & $-$0.215 & 0.103 & 0.037 \\
\addlinespace
\textbf{Screening frequency} (ref: Annual) & & & \\
\quad Biennial (every 2 years) & $-$0.412 & 0.098 & $<$0.001 \\
\quad Triennial (every 3 years) & $-$0.687 & 0.112 & $<$0.001 \\
\addlinespace
\textbf{Out-of-pocket cost} (per \$100) & $-$0.583 & 0.067 & $<$0.001 \\
\addlinespace
\textbf{Sensitivity} (per 10 pp) & 0.842 & 0.089 & $<$0.001 \\
\addlinespace
\textbf{False-positive rate} (per 5 pp) & $-$0.376 & 0.074 & $<$0.001 \\
\addlinespace
\textbf{Waiting time} (ref: Same day) & & & \\
\quad 1 week & $-$0.324 & 0.091 & $<$0.001 \\
\quad 3 weeks & $-$0.618 & 0.105 & $<$0.001 \\
\addlinespace
\textbf{Physical discomfort} (ref: None) & & & \\
\quad Mild & $-$0.198 & 0.069 & 0.004 \\
\quad Moderate & $-$0.456 & 0.084 & $<$0.001 \\
\addlinespace
\textbf{Opt-out constant} & $-$1.342 & 0.187 & $<$0.001 \\
\midrule
\multicolumn{4}{l}{Log-likelihood: $-$2,478.3} \\
\multicolumn{4}{l}{McFadden pseudo-$R^2$: 0.247} \\
\multicolumn{4}{l}{AIC: 4,980.6} \\
\multicolumn{4}{l}{BIC: 5,031.4} \\
\multicolumn{4}{l}{$N$ (respondents): 500; $N$ (choice observations): 3,000} \\
\bottomrule
\end{tabular}
\begin{tablenotes}
\small
\item \textit{Notes:} SE = standard error; pp = percentage points. Reference levels are mammography (method), annual (frequency), same day (waiting time), and none (discomfort). Cost is scaled per \$100; sensitivity per 10 percentage points; false-positive rate per 5 percentage points.
\end{tablenotes}
\end{threeparttable}
\end{table}

\subsection{Mixed Logit Model Results}
\label{subsec:mxl_results}

Table~\ref{tab:mxl_results} presents the results from the MXL model. The model was estimated with all non-cost parameters specified as random with normal distributions. The cost parameter was held fixed to facilitate WTP calculation. The MXL model demonstrated a substantial improvement in fit over the CL model, as indicated by a lower BIC (4,812.7 vs.\ 5,031.4) and a higher McFadden pseudo-$R^2$ (0.312 vs.\ 0.247).

The mean parameter estimates in the MXL model were qualitatively consistent with the CL model results, though several coefficients were larger in magnitude, consistent with the well-documented attenuation bias of the CL model in the presence of preference heterogeneity \citep{train2009discrete}. Importantly, the standard deviation estimates for several parameters were statistically significant, providing evidence of substantial preference heterogeneity in the sample.

The standard deviation for the MRI coefficient was large and significant ($\sigma_{\text{MRI}} = 1.326$, $p < 0.001$), indicating considerable heterogeneity in preferences for MRI relative to mammography. Given the estimated mean of 1.287 and standard deviation of 1.326, approximately 17\% of the sample is estimated to have a negative preference for MRI relative to mammography, suggesting that a meaningful minority of women would prefer mammography even when MRI offers equivalent performance on other attributes. Significant heterogeneity was also observed for screening frequency ($\sigma_{\text{biennial}} = 0.894$, $p < 0.001$; $\sigma_{\text{triennial}} = 1.041$, $p < 0.001$), the false-positive rate ($\sigma_{\text{FPR}} = 0.412$, $p = 0.003$), and waiting time ($\sigma_{\text{3 weeks}} = 0.687$, $p < 0.001$). The sensitivity coefficient exhibited relatively less heterogeneity ($\sigma_{\text{sensitivity}} = 0.398$, $p = 0.012$), suggesting more homogeneous preferences for higher detection rates across the sample.

\begin{table}[htbp]
\centering
\caption{Mixed Logit Model Estimates}
\label{tab:mxl_results}
\begin{threeparttable}
\begin{tabular}{lcccccc}
\toprule
& \multicolumn{3}{c}{Mean} & \multicolumn{3}{c}{Standard Deviation} \\
\cmidrule(lr){2-4} \cmidrule(lr){5-7}
Attribute & Coeff. & SE & $p$ & Coeff. & SE & $p$ \\
\midrule
\textbf{Method} (ref: Mammography) & & & & & & \\
\quad MRI & 1.287 & 0.198 & $<$0.001 & 1.326 & 0.214 & $<$0.001 \\
\quad Ultrasound & $-$0.298 & 0.137 & 0.030 & 0.756 & 0.168 & $<$0.001 \\
\addlinespace
\textbf{Frequency} (ref: Annual) & & & & & & \\
\quad Biennial & $-$0.534 & 0.132 & $<$0.001 & 0.894 & 0.156 & $<$0.001 \\
\quad Triennial & $-$0.891 & 0.158 & $<$0.001 & 1.041 & 0.189 & $<$0.001 \\
\addlinespace
\textbf{Cost} (per \$100) & $-$0.712 & 0.084 & $<$0.001 & \multicolumn{3}{c}{(fixed)} \\
\addlinespace
\textbf{Sensitivity} (per 10 pp) & 1.023 & 0.118 & $<$0.001 & 0.398 & 0.159 & 0.012 \\
\addlinespace
\textbf{FPR} (per 5 pp) & $-$0.487 & 0.098 & $<$0.001 & 0.412 & 0.138 & 0.003 \\
\addlinespace
\textbf{Waiting time} (ref: Same day) & & & & & & \\
\quad 1 week & $-$0.418 & 0.112 & $<$0.001 & 0.389 & 0.147 & 0.008 \\
\quad 3 weeks & $-$0.812 & 0.143 & $<$0.001 & 0.687 & 0.162 & $<$0.001 \\
\addlinespace
\textbf{Discomfort} (ref: None) & & & & & & \\
\quad Mild & $-$0.256 & 0.087 & 0.003 & 0.312 & 0.134 & 0.020 \\
\quad Moderate & $-$0.589 & 0.112 & $<$0.001 & 0.478 & 0.149 & 0.001 \\
\addlinespace
\textbf{Opt-out constant} & $-$1.687 & 0.243 & $<$0.001 & 1.234 & 0.267 & $<$0.001 \\
\midrule
\multicolumn{7}{l}{Simulated log-likelihood: $-$2,265.4} \\
\multicolumn{7}{l}{McFadden pseudo-$R^2$: 0.312} \\
\multicolumn{7}{l}{AIC: 4,574.8} \\
\multicolumn{7}{l}{BIC: 4,812.7} \\
\multicolumn{7}{l}{$N$ (respondents): 500; Halton draws: 1,000} \\
\bottomrule
\end{tabular}
\begin{tablenotes}
\small
\item \textit{Notes:} SE = standard error; pp = percentage points. All non-cost parameters specified as normally distributed random parameters. Cost is fixed (non-random). 1,000 Halton draws used for simulation.
\end{tablenotes}
\end{threeparttable}
\end{table}

\subsection{Latent Class Model Results}
\label{subsec:lc_results}

Model selection based on the BIC identified a three-class specification as optimal (BIC values: 2-class = 4,923.1; 3-class = 4,748.6; 4-class = 4,782.3; 5-class = 4,831.9). Table~\ref{tab:lc_results} presents the class-specific parameter estimates and class membership probabilities for the three-class model.

\textbf{Class 1: ``Accuracy-focused'' (40.2\%).} This was the largest class, comprising approximately two-fifths of the sample. Members of this class exhibited the strongest preference for higher sensitivity ($\beta_{\text{sensitivity}} = 1.534$, $p < 0.001$), the largest negative coefficient for the false-positive rate ($\beta_{\text{FPR}} = -0.712$, $p < 0.001$), and the strongest preference for MRI ($\beta_{\text{MRI}} = 1.876$, $p < 0.001$). Cost sensitivity was moderate ($\beta_{\text{cost}} = -0.487$, $p < 0.001$). This class appeared to prioritize the clinical performance of the screening test above other considerations. Class 1 members were more likely to have a family history of breast cancer (OR = 2.14, $p = 0.008$) and higher education levels (OR = 1.67, $p = 0.023$).

\textbf{Class 2: ``Cost-conscious'' (35.1\%).} The second-largest class was defined by a dominant cost coefficient ($\beta_{\text{cost}} = -1.246$, $p < 0.001$), which was more than twice as large in magnitude as in other classes. Members of this class also showed significant preferences for higher sensitivity and lower false-positive rates, but these preferences were substantially attenuated relative to Class 1. Preferences for screening method were not significant in this class. This class exhibited the strongest preference for less frequent screening ($\beta_{\text{triennial}} = -0.198$, $p = 0.312$, not significant), suggesting that cost-conscious women may view less frequent screening more favorably because it reduces cumulative costs. Class 2 membership was associated with lower household income (OR = 2.87, $p < 0.001$) and lack of health insurance (OR = 3.12, $p < 0.001$).

\textbf{Class 3: ``Convenience-seekers'' (24.7\%).} The smallest class was characterized by particularly strong preferences for shorter waiting times ($\beta_{\text{3 weeks}} = -1.423$, $p < 0.001$) and less physical discomfort ($\beta_{\text{moderate}} = -0.987$, $p < 0.001$). While members of this class also valued accuracy and lower costs, the magnitudes of these coefficients were smaller relative to Classes 1 and 2. This class showed the strongest preference for annual screening ($\beta_{\text{triennial}} = -1.134$, $p < 0.001$) and a strong aversion to the opt-out alternative ($\beta_{\text{opt-out}} = -2.345$, $p < 0.001$), suggesting high commitment to screening participation. Class 3 membership was associated with younger age (OR = 1.89, $p = 0.012$) and prior mammography experience (OR = 2.03, $p = 0.019$).

\begin{table}[htbp]
\centering
\caption{Latent Class Model Estimates (3-Class Solution)}
\label{tab:lc_results}
\begin{threeparttable}
\begin{tabular}{lccc}
\toprule
& Class 1 & Class 2 & Class 3 \\
& Accuracy-focused & Cost-conscious & Convenience-seekers \\
& (40.2\%) & (35.1\%) & (24.7\%) \\
\midrule
\textbf{Method} (ref: Mammography) & & & \\
\quad MRI & 1.876*** & 0.312 & 0.856** \\
\quad Ultrasound & $-$0.423** & $-$0.087 & $-$0.198 \\
\addlinespace
\textbf{Frequency} (ref: Annual) & & & \\
\quad Biennial & $-$0.356** & $-$0.134 & $-$0.567*** \\
\quad Triennial & $-$0.623*** & $-$0.198 & $-$1.134*** \\
\addlinespace
\textbf{Cost} (per \$100) & $-$0.487*** & $-$1.246*** & $-$0.398*** \\
\addlinespace
\textbf{Sensitivity} (per 10 pp) & 1.534*** & 0.578*** & 0.612*** \\
\addlinespace
\textbf{FPR} (per 5 pp) & $-$0.712*** & $-$0.287** & $-$0.198* \\
\addlinespace
\textbf{Waiting time} (ref: Same day) & & & \\
\quad 1 week & $-$0.234* & $-$0.167 & $-$0.678*** \\
\quad 3 weeks & $-$0.456*** & $-$0.312** & $-$1.423*** \\
\addlinespace
\textbf{Discomfort} (ref: None) & & & \\
\quad Mild & $-$0.145 & $-$0.098 & $-$0.412*** \\
\quad Moderate & $-$0.312** & $-$0.234* & $-$0.987*** \\
\addlinespace
\textbf{Opt-out constant} & $-$1.234*** & $-$0.867*** & $-$2.345*** \\
\midrule
\multicolumn{4}{l}{Log-likelihood: $-$2,198.7} \\
\multicolumn{4}{l}{AIC: 4,475.4} \\
\multicolumn{4}{l}{BIC: 4,748.6} \\
\bottomrule
\end{tabular}
\begin{tablenotes}
\small
\item \textit{Notes:} *** $p < 0.001$; ** $p < 0.01$; * $p < 0.05$. Class membership probabilities estimated using a multinomial logit specification with covariates. pp = percentage points. Reference levels as in Table~\ref{tab:cl_results}.
\end{tablenotes}
\end{threeparttable}
\end{table}

\subsection{Willingness-to-Pay Estimates}
\label{subsec:wtp_results}

Table~\ref{tab:wtp} presents WTP estimates derived from the CL and MXL models. All WTP values are expressed in U.S. dollars and represent the amount respondents would be willing to pay for a one-unit change in the attribute (or a move from the reference level to the specified level for categorical attributes).

From the MXL model, the WTP for MRI relative to mammography was \$180.76 (95\% CI: \$124.32--\$237.20), indicating that women were willing to pay approximately \$181 more out-of-pocket for an MRI-based screening program compared with mammography, holding all other attributes constant. The WTP for a 10-percentage-point increase in sensitivity was \$143.68 (95\% CI: \$108.41--\$178.95), equivalent to approximately \$250 for a 25-percentage-point improvement from 70\% to 95\% sensitivity. The WTP to avoid a 5-percentage-point increase in the false-positive rate was \$68.40 (95\% CI: \$41.23--\$95.57), or approximately \$80 per 5-percentage-point reduction in false-positive rates when considered in terms of improvement.

Women were willing to pay \$114.04 (95\% CI: \$71.89--\$156.19) to avoid a three-week wait relative to same-day results and \$82.72 (95\% CI: \$51.24--\$114.20) to avoid moderate physical discomfort relative to no discomfort. The WTP to maintain annual screening relative to triennial screening was \$125.14 (95\% CI: \$82.47--\$167.81).

\begin{table}[htbp]
\centering
\caption{Willingness-to-Pay Estimates from Conditional Logit and Mixed Logit Models}
\label{tab:wtp}
\begin{threeparttable}
\begin{tabular}{lcccc}
\toprule
& \multicolumn{2}{c}{Conditional Logit} & \multicolumn{2}{c}{Mixed Logit} \\
\cmidrule(lr){2-3} \cmidrule(lr){4-5}
Attribute & WTP (\$) & 95\% CI & WTP (\$) & 95\% CI \\
\midrule
\textbf{Method} (ref: Mammography) & & & & \\
\quad MRI & 179.76 & (128.14, 231.38) & 180.76 & (124.32, 237.20) \\
\quad Ultrasound & $-$36.88 & ($-$72.36, $-$1.40) & $-$41.85 & ($-$80.12, $-$3.58) \\
\addlinespace
\textbf{Frequency} (ref: Annual) & & & & \\
\quad Biennial & $-$70.67 & ($-$104.58, $-$36.76) & $-$74.97 & ($-$112.36, $-$37.58) \\
\quad Triennial & $-$117.84 & ($-$156.27, $-$79.41) & $-$125.14 & ($-$167.81, $-$82.47) \\
\addlinespace
\textbf{Sensitivity} (per 10 pp) & 144.43 & (113.52, 175.34) & 143.68 & (108.41, 178.95) \\
\addlinespace
\textbf{FPR} (per 5 pp) & $-$64.49 & ($-$90.12, $-$38.86) & $-$68.40 & ($-$95.57, $-$41.23) \\
\addlinespace
\textbf{Waiting time} (ref: Same day) & & & & \\
\quad 1 week & $-$55.57 & ($-$87.23, $-$23.91) & $-$58.71 & ($-$90.45, $-$26.97) \\
\quad 3 weeks & $-$106.00 & ($-$143.62, $-$68.38) & $-$114.04 & ($-$156.19, $-$71.89) \\
\addlinespace
\textbf{Discomfort} (ref: None) & & & & \\
\quad Mild & $-$33.96 & ($-$57.83, $-$10.09) & $-$35.96 & ($-$60.12, $-$11.80) \\
\quad Moderate & $-$78.22 & ($-$108.45, $-$47.99) & $-$82.72 & ($-$114.20, $-$51.24) \\
\bottomrule
\end{tabular}
\begin{tablenotes}
\small
\item \textit{Notes:} WTP = willingness to pay. 95\% confidence intervals calculated using the Krinsky-Robb method with 10,000 draws. Negative WTP values indicate the compensation required to accept the attribute level relative to the reference. pp = percentage points.
\end{tablenotes}
\end{threeparttable}
\end{table}

Figure~\ref{fig:wtp_plot} provides a visual summary of WTP estimates from the MXL model with 95\% confidence intervals.

\begin{figure}[htbp]
\centering
% Placeholder for WTP forest plot
\fbox{\begin{minipage}{0.85\textwidth}
\centering
\vspace{3cm}
\textit{[Figure placeholder: Forest plot of WTP estimates with 95\% CIs from the mixed logit model. Each attribute level is displayed as a point estimate with horizontal confidence interval bars. The vertical dashed line at \$0 separates positive WTP (right) from negative WTP (left).]}
\vspace{3cm}
\end{minipage}}
\caption{Willingness-to-pay estimates from the mixed logit model with 95\% confidence intervals.}
\label{fig:wtp_plot}
\end{figure}

\subsection{Relative Attribute Importance}
\label{subsec:attribute_importance}

The relative importance of each attribute was calculated based on the range of utility values across attribute levels, expressed as a percentage of the total utility range across all attributes. Based on the CL model estimates, test sensitivity was the most important attribute (23.4\%), followed by out-of-pocket cost (19.8\%), screening method (17.6\%), screening frequency (12.1\%), waiting time for results (11.5\%), physical discomfort (8.7\%), and false-positive rate (6.9\%). A coefficient plot illustrating the relative magnitudes of all estimated parameters is presented in Figure~\ref{fig:coefficient_plot}.

\begin{figure}[htbp]
\centering
% Placeholder for coefficient plot
\fbox{\begin{minipage}{0.85\textwidth}
\centering
\vspace{3cm}
\textit{[Figure placeholder: Coefficient plot showing estimated parameters from the CL model with 95\% confidence intervals. Parameters are displayed as horizontal bars ordered by magnitude, with the zero line for reference.]}
\vspace{3cm}
\end{minipage}}
\caption{Coefficient plot from the conditional logit model with 95\% confidence intervals.}
\label{fig:coefficient_plot}
\end{figure}

\subsection{Subgroup Analyses}
\label{subsec:subgroup}

Subgroup analyses were conducted by estimating separate CL models for respondents stratified by age group (40--49 vs.\ 50--74), income level (below vs.\ above median), prior screening experience (yes vs.\ no), and family history of breast cancer (yes vs.\ no). Key findings from the subgroup analyses are summarized below.

\textbf{Age.} Younger women (40--49) exhibited stronger preferences for MRI and placed greater importance on test sensitivity compared with older women (50--74). Older women, in contrast, were more sensitive to out-of-pocket costs and placed greater importance on screening frequency. These differences were statistically significant based on likelihood ratio tests comparing pooled and stratified models ($\chi^2 = 34.7$, $df = 12$, $p = 0.001$).

\textbf{Income.} Women with below-median household income exhibited cost coefficients approximately twice as large in magnitude as those of higher-income women ($\beta_{\text{cost}} = -0.892$ vs.\ $-0.423$, $p < 0.001$ for the interaction). Income-related differences in WTP were substantial, with lower-income women willing to pay approximately 40--50\% less than higher-income women for improvements in most attributes.

\textbf{Prior screening experience.} Women with prior mammography experience expressed stronger preferences for mammography relative to MRI compared with screening-na\"{i}ve women, potentially reflecting familiarity and comfort with the existing modality. Screening-na\"{i}ve women were more likely to choose the opt-out alternative (17.2\% vs.\ 9.1\%, $p < 0.001$).

\textbf{Family history.} Women with a first-degree family history of breast cancer placed significantly greater weight on test sensitivity ($\beta_{\text{sensitivity}} = 1.156$ vs.\ $0.768$, $p = 0.008$) and were less sensitive to cost ($\beta_{\text{cost}} = -0.412$ vs.\ $-0.623$, $p = 0.031$), consistent with a heightened focus on detection capability among women with elevated perceived risk.


% Discussion
\section{Discussion}
\label{sec:discussion}

\subsection{Summary of Findings}

This study employed a discrete choice experiment to estimate women's preferences for breast cancer screening attributes, calculate willingness-to-pay values, and identify preference heterogeneity across the screening-eligible population. Our findings demonstrate that all seven screening attributes---method, frequency, cost, sensitivity, false-positive rate, waiting time, and physical discomfort---significantly influence women's screening choices. Test sensitivity and out-of-pocket cost emerged as the two most important attributes, collectively accounting for over 40\% of the total utility range. The mixed logit model revealed substantial preference heterogeneity for most attributes, and the latent class analysis identified three distinct preference segments: accuracy-focused women (40\%), cost-conscious women (35\%), and convenience-seeking women (25\%).

\subsection{Comparison with Previous Literature}

Our finding that test sensitivity is the most important screening attribute is consistent with the majority of previous DCE studies on breast cancer screening. \citet{de2012labels} similarly found that sensitivity was the most valued attribute among Dutch women, and \citet{gyrd2016discrete} reported that Danish women placed high value on detection capability. The central importance of sensitivity across multiple studies and populations suggests that women's desire for an effective screening test is robust and should be a primary consideration in screening program design.

The strong positive preference for MRI over mammography observed in our study is noteworthy and extends the findings of \citet{mansfield2016stated}, who reported strong preferences for familiar screening modalities in an Australian context. Our results suggest that when women are presented with information about the attributes of different modalities---including MRI's typically higher sensitivity---they express a willingness to pay a substantial premium for MRI-based screening (\$181 relative to mammography). This finding has implications for the ongoing debate about the role of MRI in population-based screening, particularly for women with dense breast tissue \citep{comstock2020comparison, lord2007systematic}. However, the significant heterogeneity in MRI preferences (with approximately 17\% of women preferring mammography even when MRI offers equivalent attributes) suggests that a one-size-fits-all approach to screening modality may not align with the preferences of all women.

The significant cost sensitivity observed in our sample, and its variation across income levels and latent classes, is consistent with the broader literature on financial barriers to screening uptake \citep{trivedi2008effect, muhlbacher2016preferences}. The finding that 35\% of women belong to a cost-conscious class in which cost dominates other considerations underscores the importance of minimizing out-of-pocket costs for screening programs aimed at achieving high population coverage. The substantial income-related differences in WTP---with lower-income women willing to pay 40--50\% less than higher-income women---highlight the equity implications of cost-sharing arrangements for screening services and support policies that reduce or eliminate copayments for preventive screening \citep{sabatino2015effectiveness}.

Our identification of a convenience-seeking class (25\%) that places disproportionate weight on waiting time and physical discomfort adds to the growing evidence that process attributes matter to women considering screening \citep{whelehan2013review, nelson2020factors}. Previous DCE studies have generally not included these attributes, potentially underestimating their role in screening decisions. The finding that younger women were more likely to belong to this class may reflect generational differences in healthcare expectations and the lower perceived urgency of screening among women further from the peak incidence age for breast cancer.

The three-class structure identified in our latent class model shows parallels with the findings of \citet{phillips2019preferences}, who also identified three preference segments among UK women considering breast cancer screening. While the specific class definitions differ across studies, the consistent finding of discrete preference heterogeneity suggests that women's screening preferences cannot be adequately characterized by a single preference profile. The identification of systematic predictors of class membership---including age, income, education, insurance status, and family history---provides actionable information for targeting communication strategies and screening options to different population subgroups.

\subsection{Clinical Implications}

Our findings have several implications for clinical practice and screening program design. First, the primacy of test sensitivity in women's preferences provides support for clinical efforts to improve the sensitivity of breast cancer screening, including through supplemental screening for women with dense breast tissue and the adoption of advanced imaging technologies where clinically indicated \citep{berg2012detection, comstock2020comparison}. Clinicians should be aware that many women place substantial value on detection capability and may prefer screening strategies that maximize sensitivity, even at the cost of higher false-positive rates.

Second, the significant preference heterogeneity we observe argues for flexible screening programs that can accommodate diverse preference profiles. Rather than offering a single standard screening protocol, healthcare systems might consider providing women with a menu of screening options that vary in terms of modality, frequency, and other process features, supported by decision aids that help women identify the option most consistent with their individual values \citep{elwyn2012shared, stacey2017decision}. Such an approach aligns with the shared decision-making paradigm increasingly endorsed by professional organizations \citep{USPSTF2024breast, oeffinger2015breast}.

Third, the strong preferences for shorter waiting times for results suggest that investments in reducing the turnaround time for screening results---for example, through rapid reading protocols, artificial intelligence-assisted interpretation, or same-day results programs---could yield substantial patient welfare benefits. Given that women were willing to pay over \$114 to avoid a three-week wait relative to same-day results, the economic value of faster result delivery may justify the organizational investments required to achieve it.

\subsection{Policy Implications}

From a policy perspective, our WTP estimates provide useful inputs for cost-effectiveness analyses and coverage decisions. The finding that women value a 25-percentage-point improvement in sensitivity at approximately \$250 can inform assessments of the cost-effectiveness of supplemental screening technologies that offer higher sensitivity, such as MRI for women with dense breast tissue. Similarly, the WTP of \$181 for MRI relative to mammography provides a demand-side valuation that can be compared with the incremental cost of MRI-based screening to assess the welfare implications of different coverage policies.

The USPSTF's 2024 recommendation to lower the screening starting age to 40 is broadly consistent with our finding that women in the 40--49 age group express strong preferences for screening and are willing to pay for high-quality screening services \citep{USPSTF2024breast}. The relatively low opt-out rate in our sample (11.3\%) suggests that the vast majority of women in the screening-eligible age range prefer some form of screening over no screening, providing a preference-based rationale for extending organized screening to younger women.

The identification of a large cost-conscious segment (35\%) with strong sensitivity to out-of-pocket costs reinforces the importance of the Affordable Care Act's mandate for first-dollar coverage of preventive services, including breast cancer screening \citep{sabatino2015effectiveness}. Any policy changes that increase cost-sharing for screening could disproportionately affect this subgroup, potentially reducing screening uptake among the women who are most price-sensitive and may also face other barriers to healthcare access.

\subsection{Strengths and Limitations}

This study has several strengths. First, the inclusion of seven attributes spanning both clinical and process dimensions of screening provides a more comprehensive assessment of women's preferences than most previous DCE studies, which have typically focused on a narrower set of attributes. Second, the use of three complementary econometric models---CL, MXL, and LC---provides a robust characterization of both average preferences and preference heterogeneity. Third, the dual recruitment strategy (online panel and clinic-based) enhances the diversity of the sample and provides reassurance that our findings are not an artifact of a particular sampling approach. Fourth, the rigorous attention to survey design and data quality measures---including a DCE tutorial, practice task, dominance test, and attention checks---strengthens confidence in the validity of the choice data.

Several limitations should be acknowledged. First, as a stated preference study, the DCE is subject to hypothetical bias---the possibility that choices made in a hypothetical survey context may not perfectly correspond to actual behavior \citep{lancsar2017scale, hensher2015applied}. While DCEs are generally considered to have good external validity \citep{lancsar2017scale}, respondents may overstate their WTP or make different trade-offs in real-world settings where financial consequences are tangible and information is processed differently.

Second, the sample, while diverse, may not be fully representative of the U.S. population of screening-eligible women. Online panel participants tend to be more technologically literate and may differ from the general population in ways that influence screening preferences. Although clinic-based recruitment mitigated this concern, the overall sample may underrepresent women who are difficult to reach through either channel, including uninsured women, women in rural areas, and women with limited English proficiency.

Third, the attribute set, while comprehensive, is necessarily incomplete. Several factors that may influence real-world screening decisions---including physician recommendation, brand or institutional reputation, geographic accessibility, and the availability of results interpretation services---were not included in the DCE to maintain cognitive feasibility. The omission of these attributes means that our model captures only a subset of the factors that drive screening choices.

Fourth, the interpretation of the opt-out alternative requires caution. Women who chose ``no screening'' may have done so for diverse reasons---genuine preference for non-participation, dissatisfaction with the specific attribute levels presented, or task simplification strategies---and the opt-out constant may conflate these distinct motivations \citep{campbell2018including}.

Fifth, while the latent class model identifies discrete preference segments, the assignment of individuals to classes is probabilistic, and the characterization of classes is necessarily simplified. The three-class solution, while statistically preferred by the BIC, represents one possible characterization of preference heterogeneity, and alternative model specifications might yield different class structures.

\subsection{Future Research Directions}

Several directions for future research emerge from this study. First, revealed preference studies that link women's actual screening choices to the attributes of available screening options would provide a valuable complement to our stated preference findings, helping to assess the external validity of DCE-based estimates. Second, longitudinal studies examining how screening preferences evolve over time---particularly in response to personal screening experiences, changes in guidelines, and media coverage of screening controversies---would provide insights into the stability and malleability of preferences. Third, extending this research to diverse populations, including women from different cultural backgrounds, women with varying levels of health literacy, and women in low- and middle-income countries, would enhance the generalizability of findings and inform the design of culturally appropriate screening programs. Fourth, methodological work comparing WTP estimates from DCEs with those derived from real-world willingness-to-pay elicitation mechanisms would advance the field's understanding of hypothetical bias in health-related stated preference research. Finally, the incorporation of additional attributes---such as the availability of genetic risk assessment, the use of artificial intelligence in image interpretation, and the option for telemedicine-based result delivery---would ensure that future DCE studies capture the evolving landscape of breast cancer screening technology.


% Conclusion
\section{Conclusion}
\label{sec:conclusion}

This study contributes to the growing body of evidence on patient preferences for breast cancer screening by providing a comprehensive assessment of how women value multiple dimensions of screening programs. Using a discrete choice experiment administered to 500 women aged 40--74, we find that test sensitivity and out-of-pocket cost are the most important attributes driving screening choices, followed by screening method, frequency, waiting time for results, false-positive rate, and physical discomfort. Women express a willingness to pay substantial amounts for improvements in screening attributes, including approximately \$181 for MRI over mammography, \$250 for a 25-percentage-point increase in sensitivity, and \$114 to avoid a three-week wait for results compared with same-day delivery.

A key finding of this study is the substantial heterogeneity in screening preferences across the population. Our latent class analysis identifies three distinct preference segments---accuracy-focused, cost-conscious, and convenience-seeking women---each characterized by a different pattern of attribute trade-offs. This heterogeneity implies that a uniform screening approach is unlikely to align with the preferences of all women and supports the development of more flexible, patient-centered screening programs that offer meaningful choice among screening options.

From a policy standpoint, our results support several recommendations. First, screening programs should prioritize maintaining high test sensitivity, as this is the attribute women value most. Second, out-of-pocket costs for screening should be minimized to ensure equitable access, given the large cost-conscious segment identified in our sample. Third, investments in reducing result turnaround times and improving the comfort of screening procedures could enhance patient satisfaction and willingness to participate. Fourth, the significant preference for MRI among a substantial proportion of women, coupled with the high WTP for this modality, suggests that coverage policies for MRI screening merit reconsideration, particularly for women with dense breast tissue or elevated risk profiles.

As breast cancer screening guidelines continue to evolve and new technologies reshape the screening landscape, incorporating patient preferences into policy decisions will be essential for designing programs that are not only clinically effective but also aligned with the values and priorities of the women they serve. The quantitative preference estimates provided by this study offer a foundation for evidence-based, patient-centered screening policy.


% References
\newpage
\bibliographystyle{apalike}
\bibliography{../literature/references}

\end{document}
