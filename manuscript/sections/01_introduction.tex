\section{Introduction}
\label{sec:introduction}

Breast cancer remains the most commonly diagnosed malignancy among women worldwide, accounting for approximately 2.3 million new cases and 685,000 deaths annually \citep{sung2021global, WHO2024breast}. In the United States alone, the American Cancer Society estimates that over 310,000 new cases of invasive breast cancer will be diagnosed in 2024, with approximately 42,000 women expected to die from the disease \citep{siegel2024cancer}. The lifetime risk of developing breast cancer for American women is approximately one in eight, making it a pervasive public health concern that touches virtually every community \citep{ACS2024facts}. Despite significant advances in treatment modalities over the past several decades, early detection through systematic screening remains one of the most effective strategies for reducing breast cancer mortality, as tumors identified at earlier stages are associated with substantially better prognoses and less aggressive treatment requirements \citep{berry2005effect, tabar2003mammography}.

The centrality of screening to breast cancer control has motivated the development of organized screening programs across most high-income countries. Mammography has served as the primary screening modality for over four decades and is widely regarded as the gold standard for population-based breast cancer detection \citep{marmot2013benefits, nelson2009screening}. Randomized controlled trials conducted throughout the 1970s and 1980s demonstrated that regular mammographic screening can reduce breast cancer mortality by 15--30\% among women aged 50--74 \citep{nystrom2002long, gotzsche2013screening}. More recently, emerging technologies have expanded the screening landscape. Digital breast tomosynthesis (3D mammography) has shown improved cancer detection rates and reduced recall rates compared with conventional digital mammography \citep{friedewald2014breast}. Magnetic resonance imaging (MRI) offers superior sensitivity, particularly for women with dense breast tissue or elevated genetic risk, though at considerably higher cost and with greater rates of false-positive findings \citep{lord2007systematic, saslow2007american}. Supplemental ultrasound screening has also been investigated as an adjunct modality for women with dense breast tissue, demonstrating the ability to detect additional cancers missed by mammography alone \citep{berg2012detection, ohuchi2016sensitivity}.

Despite the demonstrated benefits of breast cancer screening, the optimal screening strategy remains a subject of vigorous scientific and policy debate. Screening guidelines issued by major medical organizations have diverged in their recommendations regarding the age at which screening should commence, the frequency of screening, and the modalities to be employed. The United States Preventive Services Task Force (USPSTF) issued a landmark updated recommendation in 2024, lowering the recommended starting age for biennial mammographic screening from 50 to 40 years for women at average risk \citep{USPSTF2024breast}. This revision represented a significant departure from the USPSTF's 2016 recommendation, which had designated screening for women aged 40--49 as an individual decision to be made in consultation with a clinician \citep{siu2016screening}. The American Cancer Society, by contrast, has recommended annual mammographic screening beginning at age 45 since 2015, with the option to begin at age 40 and transition to biennial screening at age 55 \citep{oeffinger2015breast}. International guidelines exhibit even greater variation: the United Kingdom's National Health Service (NHS) Breast Screening Programme invites women aged 50--70 for triennial mammographic screening \citep{marmot2013benefits}, while the European Commission Initiative on Breast Cancer has recommended biennial screening for women aged 50--69 with consideration of extension to ages 45--74 \citep{defined_europerecs2022}.

Underlying these guideline disagreements are fundamental tensions between the benefits and harms of screening. While mammographic screening can detect cancers at earlier, more treatable stages, it also carries well-documented risks including false-positive results that lead to unnecessary biopsies and psychological distress, radiation exposure from repeated imaging, and---most controversially---overdiagnosis \citep{welch2016breast, bleyer2012effect}. Overdiagnosis refers to the detection of cancers that would never have become clinically significant during a woman's lifetime, leading to overtreatment with its attendant physical, psychological, and financial costs \citep{puliti2012overdiagnosis, jorgensen2009overdiagnosis}. Estimates of the magnitude of overdiagnosis in mammographic screening vary widely, from less than 5\% to over 50\% depending on the methodology employed, contributing to ongoing disagreement about the net benefit of screening in certain age groups \citep{biesheuvel2007policing, etzioni2013overdiagnosis}. These trade-offs between benefits and harms underscore the fact that breast cancer screening decisions are inherently preference-sensitive---that is, the optimal choice depends in part on how individual women weigh the potential advantages against the potential disadvantages of screening.

The recognition that screening decisions are preference-sensitive has coincided with a broader movement toward patient-centered care in medicine. Shared decision-making, in which clinicians and patients collaborate to make healthcare decisions that reflect both clinical evidence and patient values, has become an increasingly prominent paradigm in clinical practice and health policy \citep{barry2012shared, elwyn2012shared}. Multiple professional organizations now recommend shared decision-making for breast cancer screening, particularly for women in age groups where the balance of benefits and harms is most uncertain \citep{USPSTF2024breast, oeffinger2015breast}. Implementing shared decision-making effectively, however, requires an understanding of what women actually value in breast cancer screening programs. While qualitative studies have explored women's attitudes toward screening \citep{hersch2013women, waller2014understanding}, there remains a need for rigorous quantitative evidence on the relative importance women place on different screening attributes and how they make trade-offs among competing considerations.

Discrete choice experiments (DCEs) have emerged as the preferred stated preference methodology for eliciting and quantifying individual preferences in health economics and health services research \citep{ryan2001using, lancsar2008conducting, debbekkergrob2012dcereview}. Grounded in Lancaster's characteristics theory of value \citep{lancaster1966new} and McFadden's random utility theory \citep{mcfadden1974conditional}, DCEs present respondents with a series of hypothetical choice scenarios in which they select their preferred option from a set of alternatives described by systematically varied attributes. By observing how individuals trade off between different attribute levels, researchers can estimate the marginal utility associated with each attribute, calculate willingness-to-pay (WTP) values when a cost attribute is included, and identify heterogeneity in preferences across population subgroups \citep{train2009discrete, hensher2015applied}. The DCE methodology has been widely applied in health economics to study preferences for a diverse range of health services, treatments, and programs \citep{clark2014discrete, soekhai2019discrete}, and a growing body of literature has employed DCEs to investigate preferences for cancer screening specifically \citep{marshall2010conjoint, de2012labels}.

Although previous DCE studies have examined preferences for breast cancer screening, several important gaps persist in the literature. Many existing studies have focused on a limited set of screening attributes, frequently omitting attributes related to the screening experience such as physical discomfort and waiting time for results, which qualitative research suggests are important to women \citep{whelehan2013review, nelson2020factors}. Furthermore, the majority of prior studies have relied exclusively on conditional logit models, which assume homogeneous preferences across the sample population and impose the restrictive independence of irrelevant alternatives (IIA) assumption \citep{hausman1984specification}. Relatively few studies have employed more flexible modeling approaches such as mixed logit or latent class models that can accommodate and characterize preference heterogeneity \citep{hole2008modelling, greene2003latent}. Finally, the rapidly evolving landscape of screening guidelines---particularly the USPSTF's 2024 recommendation change---motivates renewed investigation of how women's preferences align with or diverge from expert-recommended screening strategies.

The present study addresses these gaps by conducting a comprehensive DCE to elicit women's preferences for breast cancer screening. Specifically, this study has three objectives. First, we estimate the relative importance of seven screening attributes---screening method, screening frequency, out-of-pocket cost, test sensitivity, false-positive rate, waiting time for results, and physical discomfort---using a conditional logit model. Second, we calculate WTP estimates for improvements in each attribute to provide policy-relevant valuations denominated in monetary terms. Third, we employ mixed logit and latent class models to identify and characterize preference heterogeneity, testing whether distinct subgroups of women exhibit systematically different preference patterns that may warrant differentiated screening approaches.

The remainder of this paper is organized as follows. Section~\ref{sec:literature_review} provides a comprehensive review of the relevant literature on breast cancer screening guidelines, patient preferences in cancer screening, DCE methodology, and previous DCE applications in breast cancer screening. Section~\ref{sec:methods} details the study's methodological framework, including the theoretical foundation, experimental design, survey administration, and econometric specifications. Section~\ref{sec:results} presents the empirical results from the conditional logit, mixed logit, and latent class models, along with WTP estimates and subgroup analyses. Section~\ref{sec:discussion} discusses the findings in the context of the existing literature and considers their implications for clinical practice and policy. Section~\ref{sec:conclusion} concludes with a summary of key contributions and recommendations for future research.
