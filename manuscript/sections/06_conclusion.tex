\section{Conclusion}
\label{sec:conclusion}

This study contributes to the growing body of evidence on patient preferences for breast cancer screening by providing a comprehensive assessment of how women value multiple dimensions of screening programs. Using a discrete choice experiment administered to 500 women aged 40--74, we find that test sensitivity and out-of-pocket cost are the most important attributes driving screening choices, followed by screening method, frequency, waiting time for results, false-positive rate, and physical discomfort. Women express a willingness to pay substantial amounts for improvements in screening attributes, including approximately \$181 for MRI over mammography, \$250 for a 25-percentage-point increase in sensitivity, and \$114 to avoid a three-week wait for results compared with same-day delivery.

A key finding of this study is the substantial heterogeneity in screening preferences across the population. Our latent class analysis identifies three distinct preference segments---accuracy-focused, cost-conscious, and convenience-seeking women---each characterized by a different pattern of attribute trade-offs. This heterogeneity implies that a uniform screening approach is unlikely to align with the preferences of all women and supports the development of more flexible, patient-centered screening programs that offer meaningful choice among screening options.

From a policy standpoint, our results support several recommendations. First, screening programs should prioritize maintaining high test sensitivity, as this is the attribute women value most. Second, out-of-pocket costs for screening should be minimized to ensure equitable access, given the large cost-conscious segment identified in our sample. Third, investments in reducing result turnaround times and improving the comfort of screening procedures could enhance patient satisfaction and willingness to participate. Fourth, the significant preference for MRI among a substantial proportion of women, coupled with the high WTP for this modality, suggests that coverage policies for MRI screening merit reconsideration, particularly for women with dense breast tissue or elevated risk profiles.

As breast cancer screening guidelines continue to evolve and new technologies reshape the screening landscape, incorporating patient preferences into policy decisions will be essential for designing programs that are not only clinically effective but also aligned with the values and priorities of the women they serve. The quantitative preference estimates provided by this study offer a foundation for evidence-based, patient-centered screening policy.
